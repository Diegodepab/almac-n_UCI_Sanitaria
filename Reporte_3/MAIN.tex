\documentclass[12pt, a4paper, twoside]{article}

%% Preamble
\usepackage{pdfpages}           % Para incluir PDFs
\usepackage{graphicx}           % Para gráficos
\usepackage{subfiles}           % Para manejar subarchivos
\usepackage{hyperref}           % Para enlaces
\usepackage{listings}           % Para código fuente (ajusta lenguaje)
\usepackage{verbatim}
\usepackage[backend=bibtex,style=numeric]{biblatex} % Para citas numéricas
\addbibresource{references.bib} % Cargar archivo .bib
\usepackage{url}
\usepackage{float}


\usepackage{geometry}           % Para ajustar márgenes

% Ajustes de márgenes
\geometry{
	left=3cm,       % Margen izquierdo
	right=3cm,      % Margen derecho
	top=2.5cm,      % Margen superior
	bottom=2.5cm,   % Margen inferior
	headheight=15pt, % Altura del encabezado
	twoside          % Para documentos a dos caras
}


\graphicspath{{images/}{../images/}} % Ruta para imágenes

\begin{document}
	
	%% Cover
	\includepdf[noautoscale=true, width=\paperwidth]{cover.pdf}
	
	%% Title
	\clearpage
	\setcounter{page}{1}
	\includepdf[noautoscale=true, width=\paperwidth]{title.pdf}
	
	%%%%%%%%%%%%%%%%%%%%%%%%%%%%%%%%%%%%%%%%%%%%%%%%%%%%%%%%%%%%%%%%%%%%%%%%%%%
	
	% Índice automático
	\tableofcontents
	\newpage
	
	\section{Introducción}
	
	Este documento es una continuación de las fases anteriores del proyecto, donde se desarrollaron el \textbf{diseño conceptual y lógico} de un almacén de datos basado en la información proporcionada por la \textit{Base de Datos de Investigación Colaborativa eICU} \cite{eICU2024}, y la \textbf{integración de datos} mediante un proceso de Extracción, Transformación y Carga (ETL). Para la creación de un almacén enfocado a \textbf{pacientes con patologías respiratorias}.
	
	En esta fase, se busca construir un \textbf{cubo multidimensional} utilizando los datos del almacén previamente desarrollado,	En el ámbito hospitalario, la creación de un cubo multidimensional ofrece ventajas al permitir analizar datos clínicos complejos de manera ágil. Posteriormente se realizarán consultas en \textit{MDX} (Multidimensional Expressions) sobre dicho cubo.  Permitiendo extraer información de manera eficiente al navegar por las dimensiones y medidas del cubo.

	
	Además, se describirán los problemas encontrados durante el desarrollo y las soluciones aplicadas, asegurando que las consultas generen resultados útiles para el análisis de datos.
	
	
	
	\section{Objetivos}
	
	Los objetivos principales de este informe son los siguientes:
	
	\begin{itemize}
		\item Realizar las correcciones pertinentes en el proceso ETL que permitan el desarrollo del cubo muldimensional.
		\item Implementar un cubo multidimensional adaptado para el análisis de \textbf{pacientes con patologías respiratorias}.
		\item Desarrollar al menos 8 consultas en \textit{MDX} que exploren diferentes dimensiones y hechos del cubo, aplicando funciones avanzadas cuando sea necesario.
		\item Documentar de manera clara y replicable el proceso de creación del cubo multidimensional y las consultas \textit{MDX}, proporcionando instrucciones detalladas para su ejecución.
		\item Describir los problemas encontrados durante el desarrollo del cubo y las consultas, y detallar las soluciones empleadas.
	\end{itemize}
	
	
	 
	
	\section{Modificación del almacén de datos}
	
	\begin{itemize}
		\item Se eliminaron las columnas `Age` y `PatientUnitStayID` de la tabla de pacientes.
		\item Se seleccionaron únicamente los pacientes únicos, conservando solo su primera aparición en la base de datos. Anteriormente, el número total de pacientes era 1849, pero después de aplicar esta modificación, el total es de 1841, lo que es correcto.
		\item Para el tiempo, se utilizó la cláusula `DISTINCT` y, para la información de diagnóstico, se extrajo el primer valor de la columna `DiagnosisString` (primera dimensión). La consulta SQL utilizada para obtener los datos de diagnóstico de la base de datos fue la siguiente:
		\begin{verbatim}
			SELECT 
			DiagnosisID,
			PatientUnitStayID,
			ActiveUponDischarge,
			DiagnosisOffset,
			ICD9Code,
			DiagnosisPriority,
			SUBSTRING(DiagnosisString, 1, CHARINDEX('|'
			, DiagnosisString + '|') - 1) AS FirstDiagnosis
			FROM Diagnosis;
		\end{verbatim}
		\item Se corrigió la creación de las tablas intermedias para las relaciones `NxM`. Anteriormente, se utilizaba como origen de datos la tabla `Diagnosis` de la base de datos en lugar de la del `Data Warehouse`. Posteriormente, se realizaba un `lookup` con `IngresoUCI` utilizando el `patientID\_og`.
		\item Se realizó una transformación en la columna `Age`, cambiando su tipo de dato de `string` a `integer` para permitir su uso en consultas. Durante este proceso, se transformó el valor `'> 89'` a `90` por defecto. La consulta SQL utilizada para obtener los datos de la tabla `Patient` en la base de datos fue la siguiente:
		\begin{verbatim}
			SELECT 
			T.TiempoID,
			P.UniquePID,
			H.HospitalID,
			I.HospitalDischargeOffset,
			I.PatientHealthSystemStayID,
			I.PatientUnitStayID,
			CASE 
			WHEN I.Age = '> 89' THEN 90                      
			-- Cambiar "> 89" por 90
			WHEN ISNUMERIC(I.Age) = 1 THEN CAST(I.Age AS INT) 
			-- Convertir valores numéricos a INT
			ELSE NULL                                       
			-- Manejar otros casos como NULL
			END AS AgeInt
			FROM [eICU Collaborative Research Database].dbo.Patient I
			LEFT JOIN prueba.dbo.Paciente P ON I.uniquePID = P.uniquePID_og
			LEFT JOIN prueba.dbo.Hospital H ON I.HospitalID = H.hospitalID_og
			LEFT JOIN prueba.dbo.Tiempo T 
			ON I.HospitalDischargeYear = T.HospitalDischargeYear 
			AND I.HospitalDischargeTime24 = T.HospitalDischargeTime24;
		\end{verbatim}
	\end{itemize}
		
		
	
	\section{Creación del cubo}
	
	Para la creación del cubo, iniciaremos el proceso estableciendo la conexión con nuestra base de datos:
	
	\begin{figure}[H]
		\centering
		\includegraphics[width=0.5\textwidth]{image/origenDatos}
		\caption{Origen de datos}
		\label{fig:1}
	\end{figure}
	
	A continuación, procederemos a establecer la conexión con la base de datos, donde especificaremos el proveedor a utilizar, como se muestra en la Figura \ref{fig:2}. Dado que nuestro servidor se encuentra en el entorno local, se debe indicar únicamente . como la dirección del servidor. Es crucial, \textbf{y se debe prestar especial atención}, que la autenticación se configure utilizando SQL Server Authentication. Esto permitirá especificar el nombre de usuario como sa y la contraseña asociada, que en este caso será Almacenes. Finalmente, seleccionaremos la base de datos correspondiente, que en este escenario es prueba, como se ilustra en la Figura \ref{fig:2}.
	
	\begin{figure}[H]
		\centering
		\includegraphics[width=0.7\textwidth]{image/conexion}
		\caption{Conexión de datos}
		\label{fig:2}
	\end{figure}
	
	Llegados a este punto daremos click derecho en vistas del origen de datos. 
	
	
	 \subsection{Instrucciones para desplegar el proyecto en el equipo}
	 
	 Para ejecutar la tarea, será suficiente con descomprimir el archivo. A continuación, se deberá hacer clic derecho sobre el proyecto multidimensional y seleccionar la opción ``Propiedades''. Una vez en el apartado de propiedades, se procederá a acceder a la sección de implementación, donde se deberá seleccionar el servidor correspondiente. En este caso, se podrá elegir como servidor `localhost` o bien `localhost\textbackslash nuestroServidorSQL`. En mi situación particular, dado que disponía de dos instancias de `MSSQLSERVER`, fue necesario especificar que se utilizaría la instancia `MSSQLSERVER2`, que es la que posee la funcionalidad multidimensional.
	 
	 \begin{figure}[H]
	 	\centering
	 	\includegraphics[width=1\textwidth]{image/despliegue}
	 	\caption{Configuración de despliegue}
	 	\label{fig:3}
	 \end{figure}
	 
	 Una vez configurado todo correctamente, se procederá a hacer clic en ``Iniciar'' y el proceso se completará de manera satisfactoria. 
	 
	Posteriormente, iniciaremos los servicios de Análisis (Analysis Services) en SQL Server Management Studio y verificaremos que tanto el proyecto \textit{ProyectoMultidimensional5} como el cubo se han desplegado correctamente.
	
	 
	
	\section{Consultas en MDX}
	- Una sección con las consultas en MDX y una captura con el resultado de cada una de ellas (la imagen capturada no tiene por qué mostrar todas las tuplas resultantes). 
	
	
	\section{Instrucciones para ejecutar las consultas.}
	- Una sección con las instrucciones detalladas para que un evaluador pueda ejecutar las consultas en su máquina. 
	
	
		- Para cada consulta MDX, se muestra, además del enunciado de la consulta en sí, la consulta realizada en MDX y una captura del resultado generado.
	
	- Indica el número de consultas MDX que se han intentando, es decir, que se ha escrito código MDX y se ha mostrado el resultado de la misma, independientemente de si son correctas o no:
	
	(***) 8 o más.
	
	
	- Según tu experiencia, valora cómo están realizadas las consultas en MDX, seleccionando la opción más adecuada:
	
	(****) Todas las consultas parecen ser correctas, teniendo sentido la salida de cada una de ellas. 
	
	
	- En algunas consultas se han usado funciones más avanzada de MDX, que impliquen el uso de métodos para recorrer una jerarquía (PREVMEMBER, CURRENTMEMBER, PARENT, etc.). 

	\section{Problemas encontrados}


	Maldita maquina virtual




	- Una sección "problemas encontrados" que explique con cierto detalle los problemas que se han encontrado durante la realización de la práctica. Se permite que la información incluida en esta sección se encuentre dividida o dispersada a lo largo del documento. 

	- La sección de "problemas encontrados” (o similar) es coherente y explica con cierto detalle los problemas que se han encontrado durante la realización de la práctica (ojo, no tienen por qué haber sido resueltos). Se permite que la información incluida en esta sección se encuentre dividida o dispersada a lo largo del documento.
	

	- Calificación final.
	

	\section{Conclusión}
	
	XD


	\section{Github y conjunto de instrucciones para su correcto despliegue en SQL Server.}

	Todo el proyecto está accesible en github \cite{depab2024} donde se detalla más específicamente como desplegar en SQL.
	%%%%%%%%%%%%%%%%%%%%%%%%%%%%%%%%%%%%%%%%%%%%%%%%%%%%%%%%%%%%%%%%%%%%%%%%%%%
	\printbibliography
	
	
	%% Back Cover
	\includepdf[noautoscale=true, width=\paperwidth]{backcover.pdf}
	
\end{document}