\documentclass[12pt, a4paper, twoside]{article}

%% Preamble
\usepackage{pdfpages}           % Para incluir PDFs
\usepackage{graphicx}           % Para gráficos
\usepackage{subfiles}           % Para manejar subarchivos
\usepackage{hyperref}           % Para enlaces
\usepackage{listings}           % Para código fuente (ajusta lenguaje)
\usepackage{verbatim}
\usepackage[backend=bibtex,style=numeric]{biblatex} % Para citas numéricas
\addbibresource{references.bib} % Cargar archivo .bib
\usepackage{url}


\usepackage{geometry}           % Para ajustar márgenes

% Ajustes de márgenes
\geometry{
	left=3cm,       % Margen izquierdo
	right=3cm,      % Margen derecho
	top=2.5cm,      % Margen superior
	bottom=2.5cm,   % Margen inferior
	headheight=15pt, % Altura del encabezado
	twoside          % Para documentos a dos caras
}


\graphicspath{{images/}{../images/}} % Ruta para imágenes

\begin{document}
	
	%% Cover
	\includepdf[noautoscale=true, width=\paperwidth]{cover.pdf}
	
	%% Title
	\clearpage
	\setcounter{page}{1}
	\includepdf[noautoscale=true, width=\paperwidth]{title.pdf}
	
	%%%%%%%%%%%%%%%%%%%%%%%%%%%%%%%%%%%%%%%%%%%%%%%%%%%%%%%%%%%%%%%%%%%%%%%%%%%
	
	% Índice automático
	\tableofcontents
	\newpage
	
	% Sections
	\section{Introducción}
	
	La eICU Collaborative Research Database, fue desarrollada por el Philips eICU Research Institute (eRI) en colaboración con el MIT. Es una colección de datos desidentificados de pacientes de UCI en hospitales de Estados Unidos, que documenta detalles de sus ingresos, incluyendo diagnósticos, tratamientos, pruebas y resultados.
	En este proyecto nos centraremos en la creación de un almacén de datos para analizar información clínica de pacientes con enfermedades respiratorias en UCI. Utilizaremos una versión modelada de la base de datos eICU.
	
	
	% Sections
	\section{Objetivos}
	
	Buscamos desarrollar un diseño conceptual y lógico de un almacén de datos enfocado en los pacientes ingresados con problemas respiratorios en la UCI, utilizando una base de datos reducida del sistema eICU.
	Tras restaurar la base de datos eICU en SQL Server, realizaremos una selección de las tablas más relevantes para un almacén de datos enfocado en pacientes ingresados en la UCI. En cada tabla seleccionada, identificaremos los atributos más significativos que aporten valor al análisis clínico y seguimiento de estos pacientes. Una vez seleccionadas y estructuradas todas las dimensiones y tablas de hechos, procederemos a diseñar el modelo conceptual utilizando draw.io. Posteriormente, el modelo lógico se creará empleando un diagrama de base de datos dentro del entorno de SQL Server.
	
	
	% Sections
	\section{eICU}
	
	\section{Diseño conceptual}
	
	\section{Diseño Lógico}
	
	\begin{figure}[h!]
		\centering
		\includegraphics[width=0.8\textwidth]{image/diseño_logico.png}
		\caption{Diseño lógico}
		\label{fig:11}
	\end{figure}
	
	% Sections
	\section{Dificultades encontradas}
	
	\section{Conclusiones}

	
	%%%%%%%%%%%%%%%%%%%%%%%%%%%%%%%%%%%%%%%%%%%%%%%%%%%%%%%%%%%%%%%%%%%%%%%%%%%
	\printbibliography
	
	
	%% Back Cover
	\includepdf[noautoscale=true, width=\paperwidth]{backcover.pdf}
	
\end{document}