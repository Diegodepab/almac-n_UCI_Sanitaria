\documentclass[12pt, a4paper, twoside]{article}

%% Preamble
\usepackage{pdfpages}           % Para incluir PDFs
\usepackage{graphicx}           % Para gráficos
\usepackage{subfiles}           % Para manejar subarchivos
\usepackage{hyperref}           % Para enlaces
\usepackage{listings}           % Para código fuente (ajusta lenguaje)
\usepackage{verbatim}
\usepackage[backend=bibtex,style=numeric]{biblatex} % Para citas numéricas
\addbibresource{references.bib} % Cargar archivo .bib
\usepackage{url}


\usepackage{geometry}           % Para ajustar márgenes

% Ajustes de márgenes
\geometry{
	left=3cm,       % Margen izquierdo
	right=3cm,      % Margen derecho
	top=2.5cm,      % Margen superior
	bottom=2.5cm,   % Margen inferior
	headheight=15pt, % Altura del encabezado
	twoside          % Para documentos a dos caras
}


\graphicspath{{images/}{../images/}} % Ruta para imágenes

\begin{document}
	
	%% Cover
	\includepdf[noautoscale=true, width=\paperwidth]{cover.pdf}
	
	%% Title
	\clearpage
	\setcounter{page}{1}
	\includepdf[noautoscale=true, width=\paperwidth]{title.pdf}
	
	%%%%%%%%%%%%%%%%%%%%%%%%%%%%%%%%%%%%%%%%%%%%%%%%%%%%%%%%%%%%%%%%%%%%%%%%%%%
	
	% Índice automático
	\tableofcontents
	\newpage
	
	\section{Introducción}
	
	Este documento constituye una continuación del trabajo previo, donde se desarrolló el diseño conceptual y lógico de un almacén de datos basado en la información proporcionada por la \textit{Base de Datos de Investigación Colaborativa eICU} \cite{eICU2024}. En ese proyecto, el enfoque principal fue la selección y análisis de información relativa a \textbf{pacientes con patologías respiratorias}, determinando las tablas y columnas más relevantes para permitir un análisis exhaustivo de esta población específica.
	
	En esta nueva fase, se procederá a la implementación del proceso de \textbf{Extracción, Transformación y Carga} (ETL, por sus siglas en inglés). Este proceso es fundamental para la integración de datos en cualquier almacén de datos, ya que permite extraer datos de múltiples fuentes, transformarlos según las necesidades del modelo, y finalmente cargarlos en el sistema de almacenamiento. El proceso ETL es clave para garantizar la calidad, consistencia e integridad de los datos, factores esenciales para que el análisis posterior sea preciso y confiable.
	
	El éxito de este proceso asegura que las tablas del almacén de datos estén adecuadamente pobladas con información precisa, sentando las bases para un \textbf{análisis de datos} eficiente. Este proceso facilita la realización de consultas complejas o la integración de herramientas como \textit{Reporting Services}, que permiten la visualización clara y sencilla de la información más relevante, apoyando la toma de decisiones clínicas fundamentadas.
	
	Por lo tanto, este documento también incluirá un tutorial detallado del proceso de carga para las tablas del almacén de datos personalizado. Además, se presentará un análisis de las dificultades encontradas durante la implementación y las estrategias empleadas para superarlas.
	
	\section{Objetivos}
	
	El principal objetivo de este informe es documentar de manera detallada la ejecución del proceso ETL en dos contextos diferentes:
	
	\begin{itemize}
		\item El almacén de datos \textit{NorthwindDW}, utilizado como referencia durante las sesiones prácticas, cuya carga será replicada siguiendo los procedimientos previamente establecidos.
		\item El almacén de datos del \textit{eICU}, adaptado específicamente para el análisis de \textbf{pacientes con patologías respiratorias}, que será implementado utilizando el modelo lógico desarrollado en la fase anterior del proyecto.
	\end{itemize}
	
	A lo largo del documento se mostrará cómo se ha llevado a cabo la integración de datos en ambos almacenes, resaltando los desafíos enfrentados y las soluciones aplicadas, con el fin de proporcionar una guía clara y replicable del proceso.
	
	 
	
	\section{Modificación del almacén de datos}
	
	Durante la implementación del almacén de datos se realizaron diversas modificaciones en su estructura original, con el fin de adaptarlo lo mejor posible siguiendo las mejores prácticas para crear un buen almacén de datos. A continuación, se detallan los principales cambios efectuados y la justificación detrás de cada uno de ellos.
	
	\subsection{Ingreso a la UCI}
	El mayor cambio que hubo es que el hecho Ingreso a la UCI fue separado de la tabla paciente, Además se incorporó el atributo \texttt{hospitalDischargeOffset}, que indica la duración de la estancia del paciente en la UCI, medida en minutos desde el ingreso hasta el alta hospitalaria. Este ajuste fue sugerido para permitir el análisis del tiempo de hospitalización de cada paciente, un factor relevante en los estudios de recuperación y tratamiento de enfermedades respiratorias.
	
	\subsection{Tiempo de Alta}
	Para estudiar el tiempo en este proyecto se decidió almacenar únicamente los atributos \texttt{hospitalDischargeTime24} y \texttt{hospitalDischargeYear}, ya que la base de datos no contiene un campo \texttt{hospitalAdmitYear}, como se había sugerido originalmente. Esta decisión se tomó con base en la disponibilidad de datos y la coherencia con el diseño del modelo lógico.
	
	\subsection{Paciente}
	La tabla \textbf{Paciente}, cuyo identificador único (\texttt{PK}) es el campo \texttt{uniquePID}. Además de este identificador, se incluyeron atributos relevantes como la \texttt{edad}. También se estableció una jerarquía paralela para \textbf{género} y \textbf{etnia}, lo que permitió organizar estos atributos de forma más estructurada y lógica.
	
	\subsection{Relaciones entre Tablas}
	A continuación se describen las relaciones establecidas entre las tablas del almacén de datos, modificadas para garantizar un diseño coherente y funcional:
	
	\begin{enumerate}
		\item \textbf{Paciente} \\
		\textit{Relación: 1-n} \\
		Cada paciente puede tener múltiples ingresos a la UCI, lo que refleja que un mismo paciente puede ser readmitido en distintos momentos debido a recaídas o nuevas patologías.
		
		\item \textbf{Diagnosis} \\
		\textit{Relación: n-m} \\
		Un paciente puede tener múltiples diagnósticos asociados a un único ingreso, y el mismo diagnóstico puede repetirse en diferentes ingresos y entre distintos pacientes.
		
		\item \textbf{Medicamentos} \\
		\textit{Relación: n-m} \\
		Un ingreso puede estar asociado a la administración de varios medicamentos. Además, un medicamento puede ser utilizado en distintos ingresos de múltiples pacientes.
		
		\item \textbf{Alergia} \\
		\textit{Relación: n-m} \\
		Cada paciente puede tener múltiples alergias documentadas, las cuales son relevantes para su tratamiento durante cada ingreso a la UCI. Por lo tanto, una alergia puede estar relacionada con múltiples ingresos.
		
		\item \textbf{RespiratoryCharting} \\
		\textit{Relación: 1-n} \\
		Cada ingreso a la UCI puede tener múltiples registros de datos respiratorios, como saturación de oxígeno, volumen corriente, entre otros. Estos datos se documentan por cada ingreso individual.
		
		\item \textbf{RespiratoryCare} \\
		\textit{Relación: 1-n} \\
		Similar a \textit{RespiratoryCharting}, cada ingreso puede tener múltiples intervenciones de cuidado respiratorio, como la administración de oxígeno o ventilación mecánica, asociadas a un ingreso específico.
		
		\item \textbf{apacheApsVar} \\
		\textit{Relación: 1-1} \\
		Cada ingreso a la UCI tiene una única evaluación APS asociada. Dependiendo del diseño, esta relación puede ser de uno a uno (si se almacena como un único resumen por ingreso) o de uno a muchos (si se almacena como varios componentes individuales evaluados), se decidió que la primera forma otorga más información y mayor relación con el hecho.
		
		\item \textbf{Admissiondx} \\
		\textit{Relación: 1-n} \\
		Cada ingreso tiene un diagnóstico primario de admisión, aunque pueden existir diagnósticos  más concretos que se documentan en otra tabla, como \textit{Diagnosis} que contiene información más clara y estudiada.
		
		\item \textbf{Hospital} \\
		\textit{Relación: 1-n} \\
		Cada hospital puede tener múltiples ingresos a la UCI. Los ingresos se asocian exclusivamente a un hospital, dependiendo del centro de atención en el que se encuentre la unidad.
	\end{enumerate}
	
	Los cambios en las relaciones entre tablas buscan optimizar el diseño del almacén de datos, adaptándolo a las necesidades específicas del análisis clínico de los pacientes con patologías respiratorias. Estas modificaciones garantizan la integridad de los datos y la flexibilidad para realizar análisis detallados en contextos hospitalarios. 
	
	Para simplificar toda la información previa se puede ver el diagrama de la figura \ref{fig:11} en la cual se llega a apreciar el proyecto actualizado en el cual se basará para hacer el ETL referente a elCU
	
	
	\begin{figure}[h!]
		\centering
		\includegraphics[width=1\textwidth]{image/diseño_logico.png}
		\caption{Diagrama actualizado (Correcciones implementadas)}
		\label{fig:11}
	\end{figure}
	
	
	
	Tengo que realizar unos ultimos cambios en el diagrama y añadirlo aquí
	
	\section{ETL del almacén de datos NorthwindDW}
	
	Sección de muestre la imagen de la carga completa de NorthwindDW, con todos los "ticks" en verde. Esto querrá decir que se ha cargado correctamente el almacén. Si habéis tenido alguna dificultad o comentario a realizar sobre el ETL de este almacén de datos debéis incluirlo en esa sección. 
	
	\section{ETL del almacén de datos de pacientes con patologías respiratorias}
	
	ección que describa a modo de tutorial el proceso para alimentar las tablas del almacén de vuestro proyecto. El estilo del tutorial deberá ser similar al suministrado en las prácticas guiadas. Por cada tabla alimentada se recomienda poner la imagen con los "ticks" en verde de su correcto funcionamiento. Además, se debe incluir una imagen de la carga completa como en la sección anterior. En la descripción de esta sección deberéis comentar principalmente las dificultades encontradas y cómo lo habéis solventado.
	
	Los IDs de las tablas destino (DW) deberán ser autogenerados y no se ha usado el id obtenido de la tabla origen (salvo justificación que implique lo contrario).
	
	% Sections
	\section{Dificultades encontradas}
	
	
	Una de las principales dificultades fue la complejidad de la base de datos eICU, que incluye una gran cantidad de tablas y atributos. Esto exigió un análisis detallado para identificar las tablas y campos clave en un modelo centrado en pacientes con enfermedades respiratorias. Además, enfrentamos problemas de permisos al intentar visualizar el modelo relacional en SQL Server, lo que requirió modificar las autorizaciones del propietario de la base de datos para acceder a los diagramas de relación.
	
	Además el comienzo del trabajo podría ser lo más angustioso, al tener tanta información y opciones llega a ser un poco abrumador, desde la selección de una población concreta y modelar un almacén para dicha población termina dejando muchas dudas sobre cuantas tablas es esperable eliminar, si se esta simplificando de más o se esta tomando una decisión que afectará los siguientes apartados. 
	
	\section{Conclusiones}
	
	


	\section{Github y conjunto de instrucciones para su correcto despliegue en SQL Server.}

	Todo el proyecto está accesible en github \cite{depab2024} donde se detalla más específicamente como desplegar en SQL.
	%%%%%%%%%%%%%%%%%%%%%%%%%%%%%%%%%%%%%%%%%%%%%%%%%%%%%%%%%%%%%%%%%%%%%%%%%%%
	\printbibliography
	
	
	%% Back Cover
	\includepdf[noautoscale=true, width=\paperwidth]{backcover.pdf}
	
\end{document}